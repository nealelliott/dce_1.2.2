%
% @OSF_COPYRIGHT@
% COPYRIGHT NOTICE
% Copyright (c) 1993 Open Software Foundation, Inc.
% ALL RIGHTS RESERVED (DCE).  See the file named COPYRIGHT.DCE in the
% src directory for the full copyright text.
%
%
% (C) Copyright 1993 Transarc Corporation.
% All Rights Reserved.
%
%
% HISTORY
% $Log: adversemove,v $
% $EndLog$
%

\documentstyle[makeidx]{trarticle}
\begin{document}
\input /afs/transarc.com/project/fs/dev/de/doc/src/common/basedefs.tex

\title{ Testing the CM Under Adverse Fileset Moves }
\documentname{ Testing CM }
\confidential{OSF Members Only}
\author{Transarc Corporation} %%  \\ \ \\ OSF Members Only}
\date{15 June, 1993 \\
\copyright \number\year\ Transarc Corporation\\
All Rights Reserved}
\maketitle
\tableofcontents

\pagebreak
\section{Background}
\label{overviewref}

Moving a fileset between two file servers (or even moving between two the 
aggregates within the same file server)
is a very complicated operation. It requires a lot of coordinating efforts
between various components of the DFS system such as  
the source and the target file server, all participating Cache Managers (CMs), 
ftservers, flservers that manage FLDB, and the fts program itself. 
Therefore, a failure from any one of the DFS components involved 
in a fileset move not only would stop the move but also might have left the 
DFS system in an inconsistent state. Even worse, for files in the moving
fileset, the CM might have tokens granted from both the source 
file server and target file server. 

The recent enhanced work on the fileset move in the CM is to make the CM 
system more robust and more bulletproof under any circumstances.
Thus, in the event of an adverse fileset move, the CM will still be able to 
recover itself from failures incurred in a fileset move.

The purpose of this memo is to describe and design a set of automated test 
programs to be used to test the CM system under adverse fileset moves. 
These test programs will be used as part of regressional tests for fileset
moves. 

For detailed information about moving a fileset, please refer to available
documents {\bf DCE DFS: Token Management Under Adverse Conditions } and 
{\bf Moving Filesets Despite Errors}.
In addition, the following paragraphs also
briefly explain how the CM works during the course of a fileset move.

Before accessing any file in a fileset, the CM always requests a HERE token 
first for that fileset. The CM only requests the HERE token once for a given
fileset and the HERE token information is kept in {\bf cm_volume} data
structure representing that fileset. In essence, the HERE token is a fileset's
property and does not belong to any files. 

When the CM requests a token for a particular file, it always checks whether
it has acquired a HERE token for the fileset containing that file.
Normally, the CM would expect it already has the HERE token: the CM acquired
the HERE token when it first crossed the mount point of that fileset.

The file server uses revoking a HERE token as a way to tell the CM that a 
fileset is to be moved. Therefore, upon receiving a HERE token revocation
request for a particular fileset, the CM immediately starts
the TSR-MOVE procedure to renew appropriate tokens from the target file 
server. That is, for each token granted from the old (source) file server for 
a particular active file (i.e., the file is not closed),
the CM will get the same set of tokens from the target file server
and also send the original one back to the old server. The CM then returns
the HERE token to the old file server and concludes the move. 

Depending on state transitions of a moving fileset, 
any file operation to that fileset during a move would get an appropriate 
status returned from the file server. The returned fileset status
indicates whether a fileset is already moved to a new location, or a fileset 
is made off line for the time being, or the fileset is not in service any 
more, etc. The CM would react accordingly. 
Mostly important, the whole move operation is transparent 
to user applications: they would not even 
notice the underlying move. 

However things may not always work as planed. There could be several abnormal
situations in which a fileset move could generate some unexpected results.
Without proper recovery procedures in the CM, the whole DFS system could 
become chaos. 
For example, 
when a fileset is moved the CM may not get a HERE token revocation request 
from the file server due to a network partition or other unknown reasons. 
As a result, the CM may end up having tokens for a given file from both 
old server and new server. 
In some case, the CM may notice that it does not have a HERE token 
while requesting a regular type of token for a file in that fileset. 
The CM could have just missed a fileset move. 

\pagebreak
\section{Goals}

Our goals are pretty straightforward. We plan to write some test programs 
to better test the CM to make sure that CM will behave appropriately under
any adverse move situations. To be more specific, the goals are:
\begin{itemize}
\item to come up a set of test scenarios that will have a good coverage
in testing the CM under {\bf adverse fileset moves}. 
\item to implement a set of realistic test programs based on scenarios 
mentioned above. More important, we plan to make them automated test programs
by writing them in ITL test scripts. So that these test
programs can be integrated as part of our regressional test suite
for each future DFS release or build. 
\end{itemize}

Note that these test programs in question are only to test the CM system under
{\bf adverse fileset moves}. Implementing additional test programs to test the
CM system under {\bf normal move circumstances} is not in the scope of this 
task. 
The reason is that there are already several ITL test scripts written 
for testing the fileset moves.  
However, Section 4 will briefly
describe these test programs for testing fileset moves under normal 
circumstances. 

\section {Designs of Test Programs -- Under Adverse Fileset Moves}

As mentioned before, all test programs will be implemented in such a way
that they can be run automatedly. This can be achieved by writing these
test programs with the ITL scripts. The ITL test tool also provides a 
distributed environment in which test programs can be synchronized with each 
other. 

In addition, each test program also uses the {\bf bomb point} library 
functions to simulate a special event. For example, in order to create 
an RPC {\bf communication failure} event for the CM, a {\bf bomb point} is 
implemented in a function (of the CM source code) where each returning RPC call
is checked. When the {\tt bomb point} is set off by the {\bf bomb program}, 
the function will return {\bf rpc_s_comm_failure} as if the RPC call really 
returned a communication failure to the CM. 

Note that for more detailed information about the {\tt bomb point} and the 
ITL test tool, please refer to the related documentation.

All test programs will be running in a two-machine cell environment. The
first machine, M1, is configured as primary one running DCE server, DFS server
and DFS client (CM1). The second machine, M2, is configured as a DCE client
running both DFS server and DFS client (CM2).

\pagebreak 
\subsection {Test Program 1 -- Moving A Fileset When Network Partitioned}

\subsubsection {Test Description}

When one of the clients, CM1, is reading a file
from a particular fileset, {\tt fset1}, the network is suddenly partitioned 
between CM1 and the file server that exports {\tt fset1}. 
CM1's subsequent file operations to that fileset will encounter 
RPC communication failures. As a result, 
the HERE-token-revocation request cannot be delivered to CM1 and the fileset
is eventually moved with a success. 

After the network partition gets repaired, CM1 starts the TSR-CRASH
procedure. It is during the TSR-CRASH procedure that the CM1 
detects the fileset {\tt fset1} is actually moved to a new file server (M2). 
As part of the TSR-CRASH procedure, CM1 also performs the TSR-MOVE procedure
to recover/renew the tokens from the target file server. 

The purpose of this test is to make sure that if new files are created and
modified by CM2 while CM1 is isolated due to a network partition, new files
and changes will be visible to CM1 after the TSR-CRASH recovery. 
(Note that there is no conflicting token granted to either CM1 or CM2)


\subsubsection {Bomb Point Implantation}

Two bomb points will be implemented in the DFS/CM source code to mimic
a network partition.
\begin{itemize}
\item A bomb point is implemented in CM's ping-server function. When the bomb
point is set off, the ping-server calls, 
{\bf AFS_GetTime}, start returning 
{\bf rpc_s_comm_failure} errors to the CM. The CM marks the 
server {\bf down} and subsequent RPC calls will not be issued.
Note that in this test scenarios, the CM is only accessing files from one
file server, 
\item Another bomb point is placed in CM's HERE-token-revocation function. 
When the bomb point is set off, the CM, upon receiving a HERE-token-revocation
request, simply returns {\bf rpc_s_comm_failure} to the caller and
{\b does not} perform the TSR-MOVE procedure. 
\end{itemize}

\pagebreak 

\subsubsection {Test Program Logics}

The test program is written with an ITL script. It has two ITL test scripts running by two 
Spokes on M1 and M2 respectively. There is a third script running by Hub on machines M1
as well. Here is the detailed steps. 
\begin{programexample}
    Hub/M1		Proc1/M1			Proc2/M2
	 	 
t0  Hub:- Initialization,
	- create a fileset, fset1 and fts mount it over /:/fset1
	- create two dirs /:/fset1/tdir1 and /:/fset1/tdir2, 

t1							ls -l /:/fset1  > /dev/null
							cat /:/fset1/* > /dev/null
							create /:/fset1/tdir1/file3
							write file3
	`						close file3
							(get all open/read/write tokens)
t2							wait for SYNC 'NETWORK_PARTITIONED'
t3							wait for SYNC 'MOVE_THE_FILESET'

t4			ls -l /:/fset1 > /dev/null 				
			cat /:/fset1/* > /dev/null
			create /:/fset1/file1
			write file1
			create /:/fset1/tdir1/file4
			write file4
			(get all open/read/write tokens)
t5			wait for SYNC 'NETWORK_PARTITIONED'

t6  Hub:-Trun on `COMM_FAILURE' Bomb Point
	- Release SYNC 'NETWORK_PARTITIONED'

t7			verify the network is partitioned
			read /:/fset1/file1 
			(experiencing a comm failure)
t8			wait for SYNC 'MOVE_THE_FILESET'
t9			wait for SYNC 'NETWORK_REPAIRED'
		
t10 Hub:-Move fileset from M1 to M2
t11	-Release SYNC 'MOVE_THE_FILESET'
t12							create a new file /:/fset1/tdir1/file5
							write /:/fset1/tdir1/file5
							create /:/fset1/file2
							write /:/fset1/file2
							rename /:/fset1/tdir1/file3 to 
							       /:/fset1/tdir2/file3
t12							wait for SYNC 'NETWORK_REPAIRED'
t13 Hub:-Unset the Bomb Point 
	-(Network repaired)
	- Release SYNC 'NETWORK_REPAIRED'
	- (CM is doing TSR work)

t15							close all files

t16			Verify the results:
			file5 created by M2 is visible by M1,  
			file5's data integrity is ok,
			file3 is indeed moved from tdir1 to tdir2,	
			file1's data integrity is ok,
			file4's data integrity is ok
\end{programexample}

\pagebreak 
\subsection {Test Program 1.A -- Moving A Fileset When Network Partitioned}

\subsubsection {Test Description}

This is similar to Test No.1 except that file1 has been modified by another
client during the network partition. After the network gets repaired, 
the user at CM1 will get an ESTALE error when trying to read the same file
again.

The purpose of this test is to make sure that CM detects that files have been modified by
some other clients. Therefore the CM has to mark those files bad.


\subsubsection {Bomb Point Implantation}

The same as Test 1.

\pagebreak
\subsubsection {Test Program Logic}

The test program is written with an ITL script. It has two ITL test scripts
running on M1 and M2 respectively. Here is the detailed steps. 

\begin{programexample}
    Hub/M1		Proc1/M1			Proc2/M2
	 	 
t0  Hub:- Initialization,
	- create a fileset, fset1 and fts mount it over /:/fset1
	- create two dirs /:/fset1/tdir1 and /:/fset1/tdir2, 

t1							ls -l /:/fset1  > /dev/null
							cat /:/fset1/* > /dev/null
							create /:/fset1/tdir1/file3
							write file3
	`						close file3
							(get all open/read/write tokens)
t2							wait for SYNC 'NETWORK_PARTITIONED'
t3							wait for SYNC 'MOVE_THE_FILESET'

t4			ls -l /:/fset1 > /dev/null 				
			cat /:/fset1/* > /dev/null
			create /:/fset1/file1
			write file1
			create /:/fset1/tdir1/file4
			write file4
			(get all open/read/write tokens)
t5			wait for SYNC 'NETWORK_PARTITIONED'

t6  Hub:-Trun on `COMM_FAILURE' Bomb Point
	- Release SYNC 'NETWORK_PARTITIONED'

t7			verify the network is partitioned
			read /:/fset1/file1 
			(experiencing a comm failure)
t8			wait for SYNC 'MOVE_THE_FILESET'
t9			wait for SYNC 'NETWORK_REPAIRED'
		
t10 Hub:-Move fileset from M1 to M2
t11	-Release SYNC 'MOVE_THE_FILESET'
t12							open  /:/fset1/tdir1/file4
							read/write /:/fset1/tdir1/file4
							close file4
							create/write /:/fset1/file2
							rename /:/fset1/tdir1/file3 to 
							       /:/fset1/tdir2/file3
t12							wait for SYNC 'NETWORK_REPAIRED'
t13 Hub:-Unset the Bomb Point 
	-(Network repaired)
	- Release SYNC 'NETWORK_REPAIRED'
	- (CM is doing TSR work)

t15							close file1, file2 and file3

t16			Verify the results:
			file1 is ESTALE
			file4 is ESTALE
			file3 is indeed moved from tdir1 to tdir2,	


\end{programexample}

\pagebreak 
\subsection {Test Program 1.B -- Moving A Fileset When Network Partitioned}

\subsubsection {Test Description}

The purpose of this test is to make sure that files are opened and accessed
by CM1 will still be accessible after the network partition gets repaired despite the
fact that thoese files are removed by other clients during an interval where the CM1 is
encountered a 'network partition' and the fileset in question is moved to a new location.
This test is essentially to test the notion of {\bf ZLC} supported by the fx server.

\subsubsection {Bomb Point Implantation}

Same as Test No.1.

\subsubsection {Test Program Logic}

The test program is written with an ITL script. It has two ITL shell scripts
running on M1 and M2 respectively. Here is the detailed steps. 

\begin{programexample}

    Hub/M1		Proc1/M1			Proc2/M2
	 	 
t0  Hub:- Initialization,
	- create a fileset, fset1 and fts mount it over /:/fset1
	- create two dirs /:/fset1/tdir1 and /:/fset1/tdir2, 

t1							ls -l /:/fset1  > /dev/null
							cat /:/fset1/* > /dev/null
							wait for SYNC 'WAIT_FOR_TO_DELETE_FILES'

t2			create 50 files under 
			/:/fset1/tdir1
			write these 50 files
			read these 50 files

t3			sleep 1 minute
			wait for SYNC 'WAIT_FOR_TO_DELETE_FILES'

t4  Hub:-Release 'WAIT_FOR_TO_DELETE_FILES'

t5							before the partition,
							delete 25 files: file1 to file25


t6			wait for SYNC 'NETWORK_PARTITIONED'
t7							wait for SYNC 'NETWORK_PARTITIONED'
							wait for SYNC 'MOVE_THE_FILESET'

t8  Hub:-Trun on `COMM_FAILURE' Bomb Point
	- Release SYNC 'NETWORK_PARTITIONED'

t9			verify the network is partitioned
			read /:/fset1/file1 
			(experiencing a comm failure)
t10			wait for SYNC 'MOVE_THE_FILESET'
t11			wait for SYNC 'NETWORK_REPAIRED'
		
t12 Hub:-Move fileset from M1 to M2
t13	-Release SYNC 'MOVE_THE_FILESET'

t14							during the partition and move
							delete another 25 files: file26 to file50

t15							wait for SYNC 'NETWORK_REPAIRED'

t16 Hub:-Unset the Bomb Point 
	-(Network repaired)
	- Release SYNC 'NETWORK_REPAIRED'
	- (CM is doing TSR work)

t17							complete

t18			CM1/Proc1 to verify the results:
			- the first 25 files are fine,		
			- the second 25 files are ESTALE

\end{programexample}

\pagebreak 
\subsection {Test Program 2 -- Moving A Fileset When CM Is Not Involved}

\subsubsection {Test Description}

In a very rare situation, a fileset could be moved to a new location 
without all of participating CMs involved. That is, the 
HERE-token-revocation request never arrives to a particular CM during the 
the move. Therefore, the CM still thinks that it has some tokens granted
from the source file server. 

The CM will not notice that the fileset is actually moved until
one of users issues an RPC call to access that fileset. The RPC call
will then trigger the CM, {\bf on the fly}, to perform the 
TSR-MOVE procedure. 
However, the whole TSR operations shall be still transparent to user 
applications. 

Note that this test scenario is different than that of Test No. 1 
described above. 
The subtle difference is that in Test No. 1, the CM experiences an RPC
communication failure while issuing an RPC call. The CM is 
actually isolated due to a disconnected network. 
However, in this test scenario, the CM is not isolated from the rest of 
DFS systems. Users in the CM do not see comm_failures from RPC calls. 

The purpose of this test is to make sure that the CM will automatically 
perform the TSR-MOVE procedure {\bf on the fly} when it notices the fileset 
is already moved while issuing an RPC call. In essence, the CM is recovering
tokens from the new file server at run time. In the meantime, user 
applications  will be able to continue their file operations as if nothing 
were happening, as long as there is no conflicting token granted from the new
file server.
 

\subsubsection {Bomb Point Implantation}

One bomb point will be implemented in the DFS/CM source code to simulate
the situation where the CM never gets a HERE-token-revocation request.
\begin{itemize}
\item Another bomb point is placed in CM's Revoke-One-Token function.
When the bomb point is set off, the CM, upon receiving a token-revocation
request, will simply give up that token by returning {\bf rpc_s_ok} to the 
caller. But, in reality, the CM still thinks it has that token.
\end{itemize}

\subsubsection {Test Program Logics}

The test program is written with an ITL script. It has two ITL shell scripts
running on M1 and M2 respectively. Here is the detailed steps.

\begin{programexample}

    Hub/M1              Proc1/M1                        Proc2/M2

t0  Hub:- Initialization,
        - create a fileset, fset1 and fts mount it over /:/fset1
        - create two dirs /:/fset1/tdir1 and /:/fset1/tdir2,

t1							ls -l /:/fset1  > /dev/null
							cat /:/fset1/* > /dev/null
							open file /:/file2
							write /:/fset1/file2

							open file /:/tdir1/file3
							write /:/fset1/tdir1/file3
							(get all read tokens)
							close file3
t2		ls -l /:/fset1 > /dev/null 				
		ls -l /:/fset1/tdir1 > /dev/null
		ls -l /:/fset1/tdir2 > /dev/null
		open /:/fset1/file1
		write /:/fset1/file1

		open /:/fset1/tdir2/file4
		write file4
		open /:/fset1/tdir2/file6
		write file6

t3		wait for SYNC 'WAIT_FOR_MISS_REVOKE'
t4		wait for SYNC 'MOVE_THE_FILESET'

t5							wait for SYNC 'WAIT_FOR_MISS_REVOKE'
t6							wait for SYNC 'MOVE_THE_FILESET'

t7  Hub:-Turn on the Bomb Point:
	 CM lossing token-revocation rpcs
	-Release SYNC 'WAIT_FOR_MISS_REVOKE'

    Hub:-Move fileset from M1 to M2
	-Release SYNC 'MOVE_THE_FILESET'

t8		wait for SYNC 'WAIT_FOR_MISS_REVOKE_OFF'

t9							(after the move )
					                open /:/fset1/tdir2/file6
					                write file6

							create /:/fset1/tdir1/file5
							write file5	
							rename tdir1/file3 tdir2/file3
t10							wait for SYNC 'WAIT_FOR_MISS_REVOKE_OFF'

t11 Hub:-Turn on the Bomb Point:
	 CM is back to normal.
	-Release SYNC 'WAIT_FOR_MISS_REVOKE_OFF'
	

t12		CM1/Proc1 to verify the result:
		- file2 created before move is visible to CM1
		- file5 created after move is visible to CM1
		- file3 is indeed removed (mv) from tdir1
		- file3 is moved to tdir2
		- have recovered READ/WRITE tokens and dirty data for file1
		- file6 should be STALE
\end{programexample}

\pagebreak 
\subsection {Test Program 2.A -- Moving A Fileset When CM Is Not Involved}

\subsubsection {Test Description}

The purpose of this test is very similar to Test No.1b except that in this
test the fileset is moved while the CM1 does not handle the TSR-MOVE work
right away. Users at CM1 later notices that the fileset is actually moved
and then starts the TSR-MOVE at run time.

This test is essentially to test the notion of 'ZLC' supported by the fx
server.

\subsubsection {Bomb Point Implantation}

The same as Test No.2.

\subsubsection {Test Program Logic}

The test program is written with an ITL script. It has two ITL test scripts
running on M1 and M2 respectively. Here is the detailed steps. 

\begin{programexample}

    Hub/M1		Proc1/M1			Proc2/M2
	 	 
t0  Hub:- Initialization,
	- create a fileset, fset1 and fts mount it over /:/fset1
	- create two dirs /:/fset1/tdir1 and /:/fset1/tdir2, 

t1							ls -l /:/fset1  > /dev/null
							cat /:/fset1/* > /dev/null

t2			create 50 files under 
			/:/fset1/tdir1
			write these 50 files
			read these 50 files

t3			wait for SYNC 'WAIT_FOR_TO_DELETE_FILES'
t4			wait for SYNC 'WAIT_FOR_MISS_REVOKE'
t5			wait for SYNC 'MOVE_THE_FILESET'


t6							wait for SYNC 'WAIT_FOR_TO_DELETE_FILES'


t7  Hub:-Release SYNC 'WAIT_FOR_TO_DELETE_FILES'

t8							delete 25 files; file1 to file25
t9							wait for SYNC 'WAIT_FOR_MISS_REVOKE'
t10							wait for SYNC 'MOVE_THE_FILESET'


t11 Hub:-Trun on Bomb Point
	- Release SYNC 'WAIT_FOR_MISS_REVOKE'

t12 Hub:-Move fileset from M1 to M2
	-Release SYNC 'MOVE_THE_FILESET'

t13							after the move,
							delete another files: file26 to file50
							wait for SYNC 'WAIT_FOR_MISS_REVOKE_OFF'

t14			wait for SYNC 'WAIT_FOR_MISS_REVOKE_OFF'

t15 Hib:-Unset the Bomb Point
	-Release SYNC 'WAIT_FOR_MISS_REVOKE_OFF'

t16			Proc1/M1 to verify the results:
			- Make sure first 25 files are accessible,
			- Make sure the second helf 25 files are stale.


\end{programexample}

\pagebreak 
\subsection {Test Program 3 -- Requesting System V Locking -- No Conflicting}

\subsubsection {Test Description}

In a very rare situation, a fileset could be moved to a new location
without all of participating CMs involved. That is, the HERE-token-revocation
request never arrives to a particular CM during the the move. Therefore, the
CM still thinks that it has some tokens granted from the source file server.
The CM will not notice that the fileset is actually moved until one of users
issues an RPC call to access that fileset. This RPC call will then trigger 
the CM to perform the TSR-MOVE procedure at run time. 
 
The test environment of this test is very similar to Test No.2 execpt this
test program is to make sure that CM is able to recover the System V locks
even under this adverse fileset move.

The purpose of this test is to test that the CM's {\bf on-the-fly}
TSR-MOVE will be able to recover system locks correctly. 

\subsubsection {Bomb Point Implantation}

The same as Test No.2.

\subsubsection {Test Program Logic}

The test program is written with an ITL script. It has two ITL test scripts
running on M1 and M2 respectively. Here is the detailed steps. 

\begin{programexample}

    Hub/M1		Proc1/M1			Proc2/M2
	 	 
t0  Hub:- Initialization,
	- create a fileset, fset1 and fts mount it over /:/fset1
	- create two dirs /:/fset1/tdir1 and /:/fset1/tdir2, 

t1							ls -l /:/fset1  > /dev/null
							cat /:/fset1/* > /dev/null
							create /:/fset1/tdir2/file2
							write-lock file2
							write file2

							create /:/fset1/tdir1/file3
							write-lock file3,
							write file3

t2							wait for WAIT_FOR_SYSTEM5_LOCK


t3			ls -l /:/fset1 > /dev/null
			ls -l /:/fset1/tdir1
			ls -l /:/fset1/tdir2
			create /:/fset1/file1
			write-lock entire file1
			write file1
			create /:/fset1/tdir1/file4
			write-lock (16k, 16k+1000) and (40k, 40k+1000)
t4			wait for WAIT_FOR_SYSTEM5_LOCK


t5							test read-lock file1 and fail
t6							wait for SYNC 'WAIT_FOR_MISS_REVOKE'
t7							wait for SYNC 'MOVE_THE_FILESET'


t8			test read-lock file2 and fail
			
t9			wait for SYNC 'WAIT_FOR_MISS_REVOKE'
t10			wait for SYNC 'MOVE_THE_FILESET'
t11			wait for SYNC 'WAIT_FOR_MISS_REVOKE_OFF'

t12  Hub:-Trun on Bomb Point
	- Release SYNC 'WAIT_FOR_MISS_REVOKE'

t13 Hub:-Move fileset from M1 to M2
t14	-Release SYNC 'MOVE_THE_FILESET'

t15							After the move, 
							create /:/fset1/tdir1/file5,
							write-lock file5 and write file5,

t16							close file2,
				
t17							wait for SYNC 'WAIT_FOR_MISS_REVOKE_OFF'
t18							sleep one minute,

t19 Hub:-Unset the Bomb Point 
	-(Network repaired)
	- Release SYNC 'WAIT_FOR_MISS_REVOKE_OFF'

t20		        CM1/Proc1 to verify the results:
			- OK to get read lock on file2,
			- File1 should be STALE,
			- File4 should be STALE,
			- Should not get lock on file3,
			- Should not get write lock on file5,


t21							CM2/Proc2 to verify results:
							-OK to get write lock on file1
							-OK to get dwrite lock on file4
							(CM1 lost file1 and file4)


\end{programexample}
   
\pagebreak 
\subsection {Test Program 3.A -- Request System V Locking -- Conflicting}

\subsubsection {Test Description}

This test is very similar to Test No.3 except that an adverse fileset move
happened when the CM1 encounters a network partition. In essence, the CM
is to recover the System V locks in a cascade-failure situation.

The CM handles the same way as it does in a normal {\bf TSR-CRASH} situation.
As part of the {\bf TSR-CRASHMOVE} policy, the CM marks ESTALE all files which
used to have System V locks prior to the fileset move. Files open after the
fileset move are fine.

\subsubsection {Bomb Point Implantation}

The same as Test 1 series.

\subsubsection {Test Program Logic}

The test program is written with an ITL script. It has two ITL test scripts
running on M1 and M2 respectively. Here is the detailed steps. 

\begin{programexample}

    Hub/M1		Proc1/M1			Proc2/M2
	 	 
t0  Hub:- Initialization,
	- create a fileset, fset1 and fts mount it over /:/fset1
	- create two dirs /:/fset1/tdir1 and /:/fset1/tdir2, 

t1							ls -l /:/fset1  > /dev/null
							cat /:/fset1/* > /dev/null
							create /:/fset1/tdir1/file3
							write file3
	`						close file3
							(get all open/read/write tokens)
							wait for WAIT_FOR_SYSTEM5_LOCK
t2							wait for SYNC 'NETWORK_PARTITIONED'
t3							wait for SYNC 'MOVE_THE_FILESET'

t4			ls -l /:/fset1 > /dev/null
			ls -l /:/fset1/tdir1
			ls -l /:/fset1/tdir2
			create /:/fset1/file1
			write-lock entire file1
			write file1
			create /:/fset1/tdir1/file4
			write-lock (16k, 16k+1000) and (40k, 40k+1000)

t5			wait for WAIT_FOR_SYSTEM5_LOCK
			test read-lock file2 and fail
			
t5			wait for SYNC 'NETWORK_PARTITIONED'

t6  Hub:-Trun on `COMM_FAILURE' Bomb Point
	- Release SYNC 'NETWORK_PARTITIONED'


t7			verify the network is partitioned
			read /:/fset1/file1 
			(experiencing a comm failure)
t8			wait for SYNC 'MOVE_THE_FILESET'
t9			wait for SYNC 'NETWORK_REPAIRED'
		
t10 Hub:-Move fileset from M1 to M2
t11	-Release SYNC 'MOVE_THE_FILESET'
t12							open  /:/fset1/tdir1/file4
							read/write /:/fset1/tdir1/file4
							close file4
							create/write /:/fset1/file2
							rename /:/fset1/tdir1/file3 to 
							       /:/fset1/tdir2/file3
t12							wait for SYNC 'NETWORK_REPAIRED'
t13 Hub:-Unset the Bomb Point 
	-(Network repaired)
	- Release SYNC 'NETWORK_REPAIRED'
	- (CM is doing TSR work)

t15							close file1, file2 and file3

t16			Verify the results:
			file1 is ESTALE
			file4 is ESTALE
			file3 is indeed moved from tdir1 to tdir2,	


\end{programexample}

\pagebreak 
\subsection {Test Program 4 -- Test the CM With Abnormal Fileset States}

\subsubsection {Test Description}

As described in memo {\bf Moving Filesets Despite Errors}, during a fileset
move, there could be cases where {\bf move state} of the fileset managed by the
ftserver and the location of the fileset are not consistent. 
For example, during a fileset move, the fileset's {\bf state } in the 
source machine is 
marked VOL_MOVE_SOURCE indicating that the fileset has moved to a new location.
However, the FLDB may still point to the source server location. 
The detailed information about these abnormal states are documented in 
{\bf Section 4.2 Error states and their transitions}.

In any event, the CM when experiencing an adverse condition like this 
should be able to break out quickly without staying in an RPC loop. 

The purpose of the following three tests is to make sure that the CM is still
robust and is able to recover from these situations.  

This particular test program is only to test the CM under the situation where 
\begin{itemize}
\item the source fileset is left with {\bf VOL_MOVE_SOURCE}, 
\item the {\tt FLDB} entry is still left pointing to the {\bf source} location.
\end{itemize}
\subsubsection {Bomb Point Implantation}

No bomb point is implanted in the kernel code. However, we need 
the {\bf fts} program to set off the bomb points implanted in the ftserver code
to establish these abnormal fileset states. 


\subsubsection {Test Program Logic}

The test program is written with an ITL script. It has two ITL test scripts
running on M1 and M2 respectively. Here is the detailed steps. 

\begin{programexample}

Sequences	M1					M2

t0		create a fileset, fset1 and fts mount it over /:/fset1
		create one file: file1
            
t1							ls -l /:/fset1
							(get all read tokens)
							** May not be needed. 
t2		ls -l /:/fset1
		(get all read tokens)
		

t3		open /:/fset1/file1
		read /:/fset1/file1

t4		SYNC POINT	
		set off fts bomb point to create 
		that abnormal move state mentioned above

		SYNC POINT
		move fset1 from M1 to M2


t5		SYNC POINT
		read /:/fset1/file1 
		should fail quickly without being
		hung there. 

\end{programexample}



\subsection {Test Program 4.A -- Test the CM With Abnormal Fileset States}

\subsubsection {Test Description}

This test is very similar to Test No.4 except that it is to test the CM 
under the situation where 
\begin{itemize}
\item the source fileset is left with {\bf VOL_MOVE_SOURCE}, 
\item the {\tt FLDB} entry is left pointing to the {\bf target} location,
\item the target fileset is left with {\bf VOL_MOVE_TARGET} and {\bf VOL_OUTOFSERVICE}.
\end{itemize}

\subsubsection {Bomb Point Implantation}

Same as Test No. 4.

\subsubsection {Test Program Logic}

The test program is written with an ITL script. It has two ITL test scripts
running on M1 and M2 respectively. Here is the detailed steps. 

\begin{programexample}

Sequences	M1					M2

t0		create a fileset, fset1 and fts mount it over /:/fset1
		create one file: file1
            
t1							ls -l /:/fset1
							(get all read tokens)
							** May not be needed. 
t2		ls -l /:/fset1
		(get all read tokens)
		

t3		open /:/fset1/file1
		read /:/fset1/file1

t4		SYNC POINT	
		set off fts bomb point to create 
		that abnormal move state mentioned above

		SYNC POINT
		move fset1 from M1 to M2


t5		SYNC POINT
		read /:/fset1/file1 
		should fail quickly without being
		hung there. 

\end{programexample}

\subsection {Test Program 4.B -- Test the CM With Abnormal Fileset States}

\subsubsection {Test Description}

This test is also very similar to Test No.4 execept that it is to test the CM 
under the situation where 
\begin{itemize}
\item
the source fileset is left with {\bf VOL_MOVE_SOURCE} and {\bf VOL_OUTOFSERVICE},
\item the {\tt FLDB} entry is left pointing to the {\bf source} location
\end{itemize}

\subsubsection {Bomb Point Implantation}

Same as Test No.4.

\subsubsection {Test Program Logic}

The test program is written with an ITL script. It has two ITL test scripts
running on M1 and M2 respectively. Here is the detailed steps. 

\begin{programexample}

Sequences	M1					M2

t0		create a fileset, fset1 and fts mount it over /:/fset1
		create one file: file1
            
t1							ls -l /:/fset1
							(get all read tokens)
							** May not be needed. 
t2		ls -l /:/fset1
		(get all read tokens)
		

t3		open /:/fset1/file1
		read /:/fset1/file1

t4		SYNC POINT	
		set off fts bomb point to create 
		that abnormal move state mentioned above

		SYNC POINT
		move fset1 from M1 to M2


t5		SYNC POINT
		read /:/fset1/file1 
		should fail quickly without being
		hung there. 

\end{programexample}


\pagebreak
\section {Fileset Moves During Normal Circumstances}
   

\subsection {Test Description}

When a fileset gets moved from one file server to another file server, 
there are several state transitions for filesets 
on both the source server and the target server. 
As a result, any file operations to the fileset during the move will 
encounter different errors or status. 
Documentation {\bf Moving Filesets Despite Errors}
illustrates a detailed list of {\bf state transitions} during the normal 
circumstances. 

However, during a fileset move under a normal circumstances, 
the CM may only encounter two types of errors when accessing either the source
fileset or the target fileset. They are 
{\bf transient fileset errors} and  {\bf persistent fileset errors}. For
a transient fileset error condition, the CM understands that the fileset
is unavailable only for a short interval. This could be because that the 
server is cloning the fileset or is temporarily taking it off line. In 
any event, the CM will prints out a message indicating the fileset is busy
and then sleeps for a while and retrys it later. 

In the case of a persistent fileset error condition, the CM notices that 
the fileset is not exported by the original file server any longer. 
The filset might 
be either deleted or moved to somewhere else. The CM will contact the 
{\bf FLDB} to find out about the fileset's new location and then determine 
whether that the fileset is permanently deleted. 

Unless the fileset is deleted, user applications
will be able to continue their file operations transparently as if 
nothing happened. The CM will take appropriate recovery procedure to 
renew tokens from the new target server. 

Testing the CM under this normal circumstances becomes very straightforward. 
Any test programs that constantly issues RPC calls against that fileset
while it is moving to a new server will cause the CM to receive these two 
types of errors mentioned above. The CM will take appropriate actions based
on the type of errors it receives. 

\subsection {Test Programs}

The existing {\bf connectathon test program} will be used to the CM 
under the normal circumstances. The test will consist of three processes
running on two CMs. One of the processes is moving a fileset from M1 to
M2. In the meantime, the other two processes are running a copy of 
{\bf connectathon test} in that fileset from each CM respectively. 

The connectathon tests will run to the end without encountering any fatal
errors. 


\end{document}
